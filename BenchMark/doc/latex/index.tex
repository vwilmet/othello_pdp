Sci\-Mark 2.\-0 (\href{http://math.nist.gov/scimark2/}{\tt http\-://math.\-nist.\-gov/scimark2/}) is a Java benchmark for scientific and numerical computing. It measures several computational kernels and reports a composite score in approximate Mflops/s. This benchmark was developed at the U\-S National Institute of Standards and Technology (N\-I\-S\-T). Part of the benchmark can also be found in the Java Grande Forum Benchmark Suite (\href{http://www.epcc.ed.ac.uk/javagrande/javag.html}{\tt http\-://www.\-epcc.\-ed.\-ac.\-uk/javagrande/javag.\-html}). This benchmark contains codes on F\-F\-T, S\-O\-R (Successive Over-\/\-Relaxation over a 2\-D grid), Monte-\/\-Carlo integration, Sparse matmult (Sparse matrix vector multiplications) and L\-U factorization. We have chosen this benchmark because the same benchmark is available both in Java and C, allowing us to compare the two languages. There are many other Java benchmarks available, see \href{http://www.epcc.ed.ac.uk/javagrande/links.html}{\tt http\-://www.\-epcc.\-ed.\-ac.\-uk/javagrande/links.\-html}.

Ce module a pour but de lancer une batterie de calcul différent afin de pousser la machine à son max. A la suite de ces calculs un indice générale est calculé, correspondant à la moyenne de tous le autre calculs, permettant de connaitre les \char`\"{}caractéritiques/puissance\char`\"{} de la machine

{\bfseries Exemple de B\-O\-N\-N\-E utilisation du module \-: } \begin{DoxyVerb}  BenchMarkResultEvent event = new BenchMarkResultEvent() {

        @Override
        public void onStart() {
              System.out.println("BenchMark démarré!");
        }

        @Override
        public void onProgress(int progress) {
              System.out.println("Pourcentage : " + progress + "%");
        }

        @Override
        public void onEnd(BenchMarkResult result) {
              System.out.println("BenchMark Terminé!");
              System.out.println(result.toString());
        }
  };

  BenchMark b = new BenchMark(event);
  b.launch();\end{DoxyVerb}
 