\documentclass[a4paper,12pt]{article} 
\usepackage[utf8x]{inputenc} 
\usepackage[french]{babel} 
\usepackage{hyperref}
\usepackage{fullpage}
\usepackage{url}
\usepackage{a4wide}


\bibliographystyle{alphaurl} 
\begin{document}

\title{Étude bibliographique et de l'existant pour le projet\\
  \textit{Réalisation d'un jeu d'Othello}}

\author{Morgane Badré\\
   Benjamin Letourneau,\\
   Vincent Wilmet,\\
   Nicolas Yvon\\}
 \date{}

\maketitle

\section{Éléments bibliographiques} 

\nocite{*}

Pour plus de lisibilité, nous avons partagé notre bibliographie en trois catégories. Une première partie contiendra les généralités sur les jeux d’accessiblité. La deuxième regroupera les algorithmes de force brute : IA basique et la dernière partie présentera les algorithmes évolutionnistes et les réseaux neuraux.


\subsection{Généralités sur les jeux d'accessibilité}

Les jeux d’accessibilité, le plus souvent en tour par tour et entre deux
joueurs, consistent à déplacer des pièces sur un plateau contenant un
nombre fini de positions. Le but étant de capturer les positions du
plateau ou bien les pièces de l'adversaire afin de gagner (Jeu d’échecs, dames,...).

\subsubsection{The evolution of strong othello program}
Cet article \cite{a} est particulièrement intéressant pour notre projet car il présente les meilleurs programmes existant du jeu Othello, par ordre chronologique, en expliquant brièvement les algorithmes/heuristiques utilisés.

\subsubsection{Strategy Guide for Reversi and Reversed Reversi}
Ce site \cite{strategy} explique les différentes règles du jeu ainsi que les heuristiques dans le but de remporter une partie d'Othello (ainsi que d'anti-Othello).

\subsubsection{Site Web : Tothello}
Ce site web \cite{4} présente deux solutions possibles pour une partie d'Othello sur une grille de 6x6, résolue grâce à un logiciel s'appelant Tothello. Il explique aussi les règles de base d'Othello et de l'anti-Othello (“Reversed Othello”).

\subsection{Algorithmes de force-brute et IA basique}

Les algorithmes de force brute permettent de calculer un résultat en
énumérant systéma- tiquement toutes les solutions possibles et vérifiant pour chaque solution si elle satisfait le résultat attendu.

\subsubsection{ProbCut: An Effective Selective Extension of the
  Alpha-Beta Algorithm}
Cet article \cite{5a} décrit l'algorithme ProbCut, une amélioration de
l'algorithme alpha-beta utilisé dans le logiciel LOGISTELLO. Cet
algorithme consiste à couper les sous-arbres inutiles à la recherche
de solution pour une partie.

\subsubsection{How Machines have Learned to Play Othello}
Cet article \cite{5b} est un tour d'horizon des techniques de
résolution des jeux d'accessibilité: fonctions d'évaluations des
positions, recherches sélectives par l'algorithme Multi-ProbCut.

\subsubsection{Improving heuristic mini-max search by supervised learning (2002)}
Cet article \cite{15} présente les algorithmes utilisés pour la
réalisation du logiciel LOGISTELLO II. Il est utile car il explique en
détail le fonctionnement du logiciel (GLEM, ProbCut, ...) et nous
permettra d'améliorer nos propres algorithmes.

\subsubsection{An Analysis of Heuristics in Othello (2004)}
Cet article \cite{13} présente les différents heuristiques utilisables
pour la résolution d'Othello ainsi que les algorithmes alpha-beta et
MiniMax (Component-wise, Coin Parity, Mobility, Stability).

\subsubsection{Site web officiel de la fédération française d'Othello}
Ce site \cite{1} présente les différents algorithmes de recherches
dans les arbres de jeu (Minimax, coupures alpha-beta..) et les
méthodes spécifiques du jeu Othello comme le tri des coups, la
recherche sélective.

\subsubsection{Compte-rendu d'une étude sur le jeu Reversi (1992)}
Les recherches \cite{6} portent sur l'élaboration de la stratégie
gagnante avec une analyse de l'Othello à grille 4x4 (dénombrements sur l'arbre,
heuristique de jeu, explication de plusieurs stratégies possibles) et
de l'Othello à grille 8x8.

\subsubsection{Site Web : Radagast - Othello}
Ce site web \cite{3} présente un logiciel implémentant le jeu Othello
et des conseils sur la réalisation d'un algorithme de résolution du
jeu.

\subsubsection{Experiments with Multi-ProbCut and a new High-Quality Evaluation Function for Othello}
Cet article \cite{8} présente les améliorations apportées
au logicel Logistello, résolvant le jeu Othello. Une
amélioration de la fonction d'évaluation et de l'algorithme
ProbCut, le Multi-ProbCut, sont décrits dans cet article.

\subsubsection{Programmation des échecs et d'autres jeux}
Cet article \cite{7} explique la démarche de résolution des
jeux d'accessibilité par l'utilisation de fonctions
d'évaluation et de l'algorithme MiniMax expliqués au travers
du jeu du pile ou face puis appliquer au jeu d'échecs.

\subsection{Algorithmes évolutionnistes et réseaux neuraux}

Les algorithmes évolutionnistes consistent à faire évoluer un ensemble de solutions par itération d’un nouvel ensemble de solution amélioré. Ils sont utilisés dans le but d’optimiser la résolution de problèmes.
Les réseaux neuraux sont des modèles de calculs basés sur les réseaux de neurones biologiques de l’être humain. Ils sont représentés par des automates.



\subsubsection{Learning of position evaluation in the game of Othello (1995)}
Cet article \cite{b} est intéressant pour nous car il présente une approche
différente de la résolution d'une partie d'Othello. Cette approche
consiste, en partant d'une IA naïve, à arriver à une IA intelligente se
développant en apprenant de ses propres erreurs et des coups du joueur
opposé.

\subsubsection{An Evaluation Function for Othello Based on  Statistics (1997)}
Cet article \cite{c} expose plus en détail le logiciel Logistello II
ainsi que les algorithmes utilisés. Il utilise aussi le principe de
l'auto-apprentissage. 

\subsubsection{Coevolutionary Temporal Difference Learning for Othello (2009)}
Cet article \cite{d} expose deux manières de rendre un programme Othello intelligent et évolutif. Ces deux méthodes sont : “Temporal Difference Learning” (TDL) et “Coevolutionary Learning” (CEL).
Il nous aidera pour améliorer le joueur naïf.

\subsubsection{Strategy acquisition for the game "Othello" based on reinforcement learning}
Cet article \cite{e} explique le fonctionnement d'un nouvel algorithme basé sur le Minimax ainsi que l'auto-apprentissage. Celui-ci nous aidera également pour l'amélioration du joueur naïf.

\subsubsection{Applications of Artificial Intelligence and Machine Learning in Othello (2009)}
Cet article \cite{11} expose une étude basée sur tous les algorithmes
existant (mini-max, alpha-beta, …) ainsi que l'IA évolutive et
effectue des comparaisons.

\subsubsection{Searching for Solutions in Games and Artificial Intelligence (1994)}
Cette thèse \cite{14} présente le principe du jeu résolu et le principe de la
recherche de solution avec une IA. Cette thèse nous sera utile pour
bien appréhender le principe/fonctionnement d'une IA. 


\subsubsection{Site de Fabien Torre, université de Lille, sur l'intelligence artificielle}
Site web \cite{2}  traitant de l'intelligence artificielle et des jeux avec une présentation des algorithmes de jeu comme le Minimax, Alpha-Beta et fonctions d'évaluation. 


\subsubsection{Discovering Complex Othello Strategies Through Evolutionary Neural Networks}
Cet article \cite{9} présente l'utilisation de réseaux neuraux et d'encodage basé sur l'ADN pour la résolution d'Othello.



\section{Étude de l'existant}

Depuis les années 80 \cite{antho} \cite{tour} \cite{otstory}, de nombreux programmes Othello français \cite{prog1} et
étrangers \cite{prog2} ont été créés et se sont défiés lors de tournois afin de
tester leurs performances. Ainsi, nous avons regroupé les programmes
les mieux classés, les plus connus mais aussi des projets autour du
jeu d'Othello.

\subsection{Programmes Othello de renommés}

\subsubsection{Iago}
IAGO \cite{iago} est probablement le premier programme Othello qui ait été créé en 1976 par l'université “Caltech” de Pasadena.
Plusieurs programmes ont porté ce même nom. En effet, en 1982, Paul
Rosembloom de l'université Carnegie-Mellon à Pittsburg (Pennsylvanie)
développe le logiciel IAGO. Ce dernier est plus abouti et plus
performant que la première version. Il est fortement centré sur le
concept de mobilité dont le principe est de capturer les pièces de l'adversaire de façon permanente et limiter les mouvements possibles pour l'adversaire.
Il a utilisé les algorithmes “Current-Mobility”, “Edge-Stability” et “Potential Mobility”.


\subsubsection{Bill}
Il y a eu quatre versions de Bill \cite{bill} dont la première date de 1985 par l'université de Carnegie Mellon et la dernière de 1990 par Kai Fu Lee et Sanjoy Manahan. 
Il est considéré comme un successeur de IAGO car leurs algorithmes sont basés sur le même concept. Sa faculté de prévision des coups suivants lui a permis de vaincre IAGO en un contre un (Inférence bayésienne).
Cette faculté est basée sur des tables contenant les combinaisons  à suivre pour une partie gagnante (killer table).

\subsubsection{Keyano}
Ce programme \cite{keyano} a été écrit par Mark Brockington pendant ses recherches de doctorat en 1996. Pendant les années suivantes, il a fait parti des meilleurs programmes au monde. 
Il utilise essentiellement l'algorithme ProbCut ainsi qu'une IA évolutive gérant une base de données qui contient les jeux de combinaisons possibles.  


\subsubsection{Logistello I}
Ce programme \cite{logistello} a été créé par Michael
Buro en 1991. Il a remporté son premier tournoi en partageant la
première place avec STELLA en Octobre 1993 en Allemagne et en battant KEYANO,  REV (renommé KITTY en 1994), NICKSREV, VERS2 et CHIGOREV. Il domine toutes les IA d'Othello jusqu'en 1996 lorsqu'il a été battu par HANNIBAL.

\subsubsection{Logistello II}
LOGISTELLO II \cite{logistello} est une évolution de Logistello I créé par Michael Buro. Il est considéré comme l'une des meilleurs IA d'Othello. Il a été le premier logiciel à battre le champion du monde Takeshi Mukarami (japonais) en 1997. En 1998, LOGISTELLO arrête les tournois.


\subsubsection{Hannibal}
HANNIBAL, par Martin Piotte et Louis Geoffroy, est un programme d'apprentissage d'Othello écrit en 1996. D'après les résultats des tournois en 1997, il aurait été à peu près aussi fort que LOGISTELLO II à ce moment là. Depuis que LOGISTELLO I a arrêté en 1998, HANNIBAL était certainement le programme actif le plus fort à la fin des années 90. HANNIBAL utilise l'algorithme Minimax et une fonction d'évaluation neurale.

\subsubsection{Zebra}
ZEBRA \cite{3} est un programme implémentant Othello en utilisant
l'algorithme Multi-ProbCut et NegaScout. Il a été créé
en 1997 et a été amélioré jusqu'en 2001. Il a été plusieurs fois en 3ème place lors de tournois.

\subsubsection{Edax}
EDAX est un programme \cite{edax} implémentant une intelligence artificelle pour Othello en utilisant les ressources des machines actuelles, CPU 64-bits et multi-coeur. Il a été créé en 1998 et a été régulièrement amélioré depuis.

\subsubsection{Cyrano}
CYRANO \cite{cyrano} est une applet implémentant une version améliorée de l'algorithme Alpha-Beta, le NegaScout. La dernière version de CYRANO date de 2005.

\subsection{Projets autour d'Othello}

\subsubsection{Projet JACOTHELLON}
A l'université de Montpellier, en deuxième année d'IUT Informatique,
un binôme composé de Audrey Colbrant – Elodie Nouguier, a développé un
jeu nommé Jacothellon \cite{jacothellon}. 
Le projet avait pour but de développer un logiciel mettant en scène deux joueurs : soit deux humains jouant l'un contre l'autre, soit un humain contre une intelligence artificielle à travers le jeu Othello. Le jeu a été développé en Java et sous Eclipse.
Ce rapport nous permet d'avoir une base et une vue d'ensemble sur les
étapes et les difficultés du développement d'un tel jeu.

\subsubsection{Projet jeu d'Othello en java}
Guillaume Sauveur, à l'école d'ingénieur en informatique l'EnsiCaen, a
réalisé un projet en trinôme en java : Othello
\cite{projothello1}. Nous avons à disposition leur compte-rendu
présentant le déroulement du projet notamment avec une partie
portant sur l'intelligence artificielle avec différents niveaux de
difficultés. L'algorithme Minimax, les coupures alpha-bêta et
la fonction d'évaluation sont utilisés pour développer l'IA.


\subsubsection{Projet : Othello en C++ et SFML}
Un rédacteur de tutoriels, Pierre-Emmanuel Mercier, a mis à
disposition sur son site Ace-Art le jeu Othello \cite{projothello2} qu'il a développé avec Alexis Bouvot qui a intégré l'IA.  
Il fournit dans une archive un diagramme de classes UML, le code source en C++ avec la librairie SMFL. 
Le diagramme et le code vont nous être utiles pour avoir une idée globale du travail à réaliser.
 



\bibliography{biblio} 

\end{document}